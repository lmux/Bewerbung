\documentclass[12pt,a4paper,oneside]{book}
% Requires Ghostscript	currfile		
\usepackage{verbatim,currfile}
\usepackage[utf8]{inputenc}     

\newcommand* \buildDir          {build/} %z.B.  './' oder 'build/'
\newcommand* \inhalt            {../Inhalte-Latex}
\newcommand* \anschreibenRef    {\inhalt/Inhalt-Anschreiben.tex}
\newcommand* \motivationsschreibenRef    {\inhalt/Inhalt-Motivationsschreiben.tex}
\newcommand* \bewerberInfos     {\inhalt/BewerberInfos.tex}
\newcommand* \cvRef             {\inhalt/Inhalt-CV.tex}
\newcommand* \anlagenRef        {\inhalt/Inhalt-Anlagen.tex}

\input{\bewerberInfos}
\usepackage{Vorlage_Bewerbung}
% !!! Requires shell-escape !!!
% http://tex.stackexchange.com/questions/37489/how-can-i-enable-write-18-on-a-miktex-installation
% just add -shell-escape to pdflatex
\RequirePackage{filecontents}

%%%%%%%%%% Anschreiben %%%%%%%%%% 
\begin{filecontents}{anschreibenGen.tex}
\documentclass[DINmtext,draft=false]{scrlttr2}
\usepackage[utf8x]{inputenc}
\usepackage{layouts,etoolbox}
\usepackage{Vorlage_Anschreiben}
\newcommand* \buildDir          {build/} %z.B.  './' oder 'build/'
\newcommand* \inhalt            {../Inhalte-Latex}
\newcommand* \anschreibenRef    {\inhalt/Inhalt-Anschreiben.tex}
\newcommand* \motivationsschreibenRef    {\inhalt/Inhalt-Motivationsschreiben.tex}
\newcommand* \bewerberInfos     {\inhalt/BewerberInfos.tex}
\newcommand* \cvRef             {\inhalt/Inhalt-CV.tex}
\newcommand* \anlagenRef        {\inhalt/Inhalt-Anlagen.tex}


\input{\bewerberInfos}
\farbe{blue}
\begin{document}
\input{\anschreibenRef}
\end{document}
\end{filecontents}
\immediate\write18{pdflatex -output-directory="\buildDir" anschreibenGen.tex}

% Filename without blanks
\newcommand* \anschreiben {\buildDir anschreibenGen.pdf}
\newcommand* \motivationsschreiben {\buildDir motivationsschreibenGen.pdf}
\newcommand* \lebenslauf {\buildDir cvGen.pdf}

%%%%%%%%%% Motivationsschreiben %%%%%%%%%% 
\begin{filecontents}{motivationsschreibenGen.tex}
	\documentclass[DINmtext,draft=false]{scrlttr2}
	\usepackage[utf8x]{inputenc}
	\usepackage{layouts,etoolbox}
	\usepackage{Vorlage_Motivationsschreiben}
	\newcommand* \buildDir          {build/} %z.B.  './' oder 'build/'
\newcommand* \inhalt            {../Inhalte-Latex}
\newcommand* \anschreibenRef    {\inhalt/Inhalt-Anschreiben.tex}
\newcommand* \motivationsschreibenRef    {\inhalt/Inhalt-Motivationsschreiben.tex}
\newcommand* \bewerberInfos     {\inhalt/BewerberInfos.tex}
\newcommand* \cvRef             {\inhalt/Inhalt-CV.tex}
\newcommand* \anlagenRef        {\inhalt/Inhalt-Anlagen.tex}

	
	\input{\bewerberInfos}
	\farbe{blue}
	\begin{document}
		\input{\motivationsschreibenRef}
	\end{document}
\end{filecontents}
\immediate\write18{pdflatex -output-directory="\buildDir" motivationsschreibenGen.tex}

%%%%%%%%%% CV %%%%%%%%%% 
\begin{filecontents}{cvGen.tex}
\PassOptionsToPackage{dvipsnames}{xcolor}
\documentclass[12pt,a4paper,sans]{moderncv} 
\usepackage[breit]{Vorlage_Lebenslauf}
\usepackage[utf8x]{inputenc}
\PrerenderUnicode{ßÄäÜüÖö}
\newcommand* \buildDir          {build/} %z.B.  './' oder 'build/'
\newcommand* \inhalt            {../Inhalte-Latex}
\newcommand* \anschreibenRef    {\inhalt/Inhalt-Anschreiben.tex}
\newcommand* \motivationsschreibenRef    {\inhalt/Inhalt-Motivationsschreiben.tex}
\newcommand* \bewerberInfos     {\inhalt/BewerberInfos.tex}
\newcommand* \cvRef             {\inhalt/Inhalt-CV.tex}
\newcommand* \anlagenRef        {\inhalt/Inhalt-Anlagen.tex}

\input{\bewerberInfos}
\moderncvcolor{blue} 
\begin{document}
\input{\cvRef}
\end{document}
\end{filecontents}
\immediate\write18{pdflatex  -output-directory="\buildDir" cvGen.tex}
% Compilieren trotz z.B. marvosym Fehler: '--interaction scrollmode' anfügen

%%%%%%%%%% Include Cmds %%%%%%%%%% 
\newcommand\anschreibenImport{%
\IfFileExists{\buildDir anschreibenGen.pdf}{
	\includepdf[pages=-, addtotoc={1, pageRef, 0,Anschreiben,anschreiben}]{\buildDir anschreibenGen}
}{
	\Huge ooops \normalsize etwas stimmt nicht. \\
	
	Wahrscheinlich ist die \textbf{\textbackslash buildDir} Variable nicht richtig gesetzt oder \textbf{Shell--Escape} wurde nicht aktiviert, siehe\\ http://tex.stackexchange.com/questions/37489/how-can-i-enable-write-18-on-a-miktex-installation
	\\\\
	\textbf{kein Anschreiben} im Pfad "\buildDir\textbackslash anschreibenGen.pdf" \hspace{1pt} gefunden. Bitte "\textbackslash buildir"\hspace{1pt}prüfen\newline\\
}%
}

\newcommand\cvImport{%
\IfFileExists{\buildDir cvGen.pdf}{
	\bookmark[page=\thepage,level=0]{Lebenslauf}
	\includepdf[pages=-]{\buildDir cvGen}
}{
	\textbf{kein CV} im Pfad "\buildDir\textbackslash cvGen.pdf" \hspace{1pt} gefunden. Bitte "\textbackslash buildir"\hspace{1pt}prüfen
}%
}

\newcommand\motivationsschreibenImport{%
	\IfFileExists{\buildDir motivationsschreibenGen.pdf}{
		\includepdf[pages=-, addtotoc={1, pageRef, 0,Motivationsschreiben,motivationsschreiben}]{\buildDir motivationsschreibenGen}
	}{
		\Huge ooops \normalsize etwas stimmt nicht. \\
		
		Wahrscheinlich ist die \textbf{\textbackslash buildDir} Variable nicht richtig gesetzt oder \textbf{Shell--Escape} wurde nicht aktiviert, siehe\\ http://tex.stackexchange.com/questions/37489/how-can-i-enable-write-18-on-a-miktex-installation
		\\\\
		\textbf{kein Motivationsschreiben} im Pfad "\buildDir\textbackslash motivationsschreibenGen.pdf" \hspace{1pt} gefunden. Bitte "\textbackslash buildir"\hspace{1pt}prüfen\newline\\
	}%
}

%%%%%%%%%% Final %%%%%%%%%%
\farbe{blue}
\begin{document}
\anschreibenImport
\cvImport
%\motivationsschreibenImport
\input{\anlagenRef}
\end{document}


%\immediate\write18{gs -sDEVICE=pdfwrite -dCompatibilityLevel=1.4 -dNOPAUSE -dQUIET -dBATCH -sOutputFile=\build \currfilebase -compressed.pdf \build \currfilebase .pdf}

%\immediate\write18{gs -sDEVICE=pdfwrite -dCompatibilityLevel=1.4 -dNOPAUSE -dBATCH -sOutputFile=build/Bewerbung_Komplett-compressed.pdf build/Bewerbung_Komplett.pdf}

% gs -sDEVICE=pdfwrite -dCompatibilityLevel=1.4 -dNOPAUSE -dBATCH -sOutputFile=build/Bewerbung_Komplett-compressed.pdf build/Bewerbung_Komplett.pdf

% "C:\Program Files\gs\gs9.07\bin\gswin64c.exe" -sDEVICE=pdfwrite -sPAPERSIZE=a4 -dCompatibilityLevel=1.6 -dNOPAUSE -dBATCH -sOutputFile=build/Bewerbung_Komplett-compressed.pdf build/Bewerbung_Komplett.pdf
