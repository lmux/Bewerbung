%\oldSymbols % Marvosymbols

\makeatletter
%%\@setplength{subjectaftervskip}{\baselineskip+5pt} % 29pt default %Abstand Betreff <-> Anrede
%%\@addtoplength{refvpos}{-8mm} %Textbereich 8mm nach oben
\@setplength{refvpos}{\useplength{toaddrvpos}} % see https://tex.stackexchange.com/questions/33278/scrlttr2-how-can-i-remove-the-white-space-between-address-and-subject-line
\makeatother 
%\setlength{\footskip}{0pt} % Abstand Anlagen Seitenende anpassen (auch negativ, wie -8pt)
% Worttrennungshilfen
%	\linebreak \mbox{} \sloppy  \showhyphens{}   \hyphenation{} Wort\-trennbeispiel
\emergencystretch=20pt\tolerance=1200\hyphenpenalty=1000% Gewichte für Trennung
%\hyphenation{Louisiana}% nicht trennen
% zeilenumbrueche mit \linebreak
% Worttrennung mit Wort\-trennung erzwingen


\begin{letter}{}%
	\opening{}
	\titel{\emph{Meine Motivation, als Frontend-Entwickler bei Ihnen einzusteigen\\}}
	\begin{wrapfigure}[14]{I}{13em}
		%\rule{5em}{4\baselineskip}% place your image here using \includegraphics
		\includegraphics[width=0.3\textwidth]{\fotomotivationsschreiben}%
	\end{wrapfigure}

	%hier folgt der erste Absatz, der auch gleichzeitig die \textbf{Einleitung} darstellt. Am besten kommt man gleich zur Sache: Warum interessiert mich diese Stelle, und warum halte ich mich für geeignet.  Kommentar \glqq Bsp\grqq .
	Sehr geehrte Frau Schmidt,\\\\
	während meines Studiums sowie aktuell engagierte ich mich zusammen mit vielen zielorientierten Menschen in verschiedenen Gremien, wie der Studierendenvertretung und Projekten, wie dem des Reudnetz w.V. %passionierten
	Die Vorstellung, nun auch beruflich an Lösungen mitzuwirken, die sich mit den Problemen unserer Zeit beschäftigen, erfüllt mich sehr.
	
	Dabei habe ich jetzt ebenfalls die Möglichkeit, meine fachlichen Kompetenzen der Informatik einzubeziehen. Bereits vor Studienbeginn hat mich insbesondere die Webentwicklung im Frontend begeistert. Eine meiner jüngsten Erinnerung hiervon ist die Modifikation eines Browser-Addons, um eine Webseite nach meinen Belieben umzufunktionieren. Über die Jahre konnte ich mithilfe des Studiums sowie in meiner ersten Berufserfahrung mir eine breite Wissensbasis zulegen und Methoden, um diese kontinuierlich zu erweitern. Meine Wissensbasis umfasst dabei zusätzlich grundlegendes Wissen zum Backend sowie zur Administrierung der Computer und Netzwerke, um den Kontext der zu entwickelnden Lösungen zu verstehen und zu berücksichtigen.
		
	Mit meinen sehr guten Englischkenntnissen haben Sie die Möglichkeit, die Kollaboration zwischen den einzelnen Teams Ihres internationalen Netzwerks in der Frontend-Entwicklung zu fördern. Zu meiner Grundschulzeit besuchte ich die Leipzig International School. Seitdem begleitet mich die englische Sprache im Alltag. Zuhause spreche ich in meiner internationalen Wohngemeinschaft ebenfalls Englisch. Zusätzlich ist für das Verständnis der Technologien, insbesondere der neueren Frontend-Frameworks, das Begreifen der in Englisch verfassten Originaldokumentation essentiell.
	
	Für weitere Informationen zu meiner Person, Motivation oder meinen Kompetenzen stehe ich jederzeit gerne zur Verfügung.
	%\par
	%Im zweiten Absatz beginnt der \textbf{Hauptteil}. Hier stellt man sich vor, und hier sollte man anhand von Qualifikationen und Erfahrungen belegen, warum man die Anforderungen erfüllt. Im Hauptteil sollte man auch persönliche Qualitäten erwähnen: Welche Hard und Soft Skills bringe ich mit (ich bin teamfähig, flexibel, etc.).
	%Ich bin ein begeisterter Programmierer mit Fokus auf den Frontend-Stack. Auf www.lmux.de sind meine aktuellen größeren Projekte aufgelistet.
	%Mein Fokus als begeisterter Softwareentwickler liegt auf dem Frontend-Stack. Auf www.lmux.de sind meine aktuellen größeren Projekte aufgelistet.
	%%Ich wäre für das Scrum-Team bestens geeignet, da ich mehrjährige Erfahrung mit den im Frontend einsehbaren und ausgeschriebenen Technologien besitze.%, welche in der Stellenausschreibung ausgewiesen und im Frontend für das Flottenmanagement einsehbar sind. %-> arbeite seit einem jahr mit leaflet 
	%Bereits vor Beginn meines Studiums habe ich mit HTML, CSS und Javascript gearbeitet. Im Informatik-Studium und Praktikum habe ich Webseiten-Prototypen auf Basis von Anforderungsspezifikationen entwickelt und dabei von dem Versionskontrollsystem Git, Design-Patterns wie MVC, SQL-Abfragen und Dokumentationswerkzeugen wie JSDoc Gebrauch gemacht.
	%Die Abfrage von SQL-Abfragen sowie die Verwendung verschiedener Datenbankmanagementsystem ist mir ebenfalls geläufig.
	 %Linux verwende ich auf all meinen Geräten, sei es PC, Laptop oder Router. Momentan arbeite ich mit dem Framework Angular an einem eigenen Projekt.%und nutze daher auch Microservices in NodeJS.
	%
	%Finde toll, dass im Frontend open-source technologien verwendet werden und der eigene Quellcode ebenfalls Open-Source lizensiert ist. -> nur ein kleines skript
	%TODO: Beispiel Teamfähigkeit
	%Die Erarbeitung von Lösungen innerhalb von Gruppen, etwa in Scrum-Meetings oder Plenumssitzungen bereitet mir große Freude. 
	%etwa in Scrum-Meetings, bereitet mir große Freude. 
	%Da ich auch grundlegende Kenntnisse über den Backend-Stack besitze, ist eine Kommunikation mit den anderen Fachbereichen ebenfalls problemlos möglich.
	%%Die eigenverantwortliche Abarbeitung von Aufgaben konnte ich zuletzt in meinem Projekt \textit{RadioMap} beweisen. In der befristeten Zeit des Projektes konnte ich alle eigens analysierten und definierten Anforderungen an die Anwendung umsetzen, %ohne Abstriche an der Dokumentation sowie dem Test der Funktionsfähigkeit zu vollziehen.
	%%mitsamt Dokumentation und Test des Gesamtsystems.

	%Meine Gehaltsvorstellung liegt bei 47.840 Euro brutto pro Jahr.	Mögliches Startdatum wäre der 01.07.2021.
	%Der letzte Absatz gehört dem \textbf{Schluss}. Hier bekundet man nocheinmal sein Interesse, sowie die Reaktion, die man sich wünscht. (Über eine Einladung zu einem persönlichen Gespräch würde ich mich sehr freuen.)

	%TODO: "ich" vermeiden
	%Ich freue mich darauf, mit Ihnen im persönlichen Gespräch zu klären, wie ich Ihrem Team den größten Nutzen, zur Erreichung des Ziels, bringen kann.
	%Über die Möglichkeit mit Ihnen im persönlichen Gespräch zu klären, wie ich Ihrem Team den größten Nutzen bringen kann, freue ich mich sehr.
	%\vfill %bei weniger als 3 zeilen platz
	%\newline%\newline%\newline 
	Mit freundlichen Grüßen\\\sig%
	%\newline 
	%\textbf{\encl{}}
	%Lebenslauf \\
	%Zeugnis 
	%\encl{: Lebenslauf, Zeugnis}
	%meine stärken: 
	%begeisterter programmierer mit fokus auf webprogrammierung, welcher in seiner freizeit webanwendungen programmiert. veröffentlich auf www.lmux.de ( mehr folgen)
	%sicherheit wichtig: teilnahme an ctfs, um schwachstellen zu entdecken 
	%kenntniss über technologien im front- und backend
	%teamarbeit/arbeit in gruppen: plenum/vereinen
	%gesamtüberblick: Zusammenarbeit mit System Engineers und Backend-Entwickler*innen
	
	%wieso dieses unternehmen:
	% 
	%ich bin softwareentwickler geworden, um Lösungen zu entwickeln, die unsere gesellschaft voranbringen
	
	%die herausforderung unserer zeit, die der klimawandel, laesst sich nur durch einen radikalen Verkehrswandel
	
	
	%was für unternehmen geil wären: open-source (software widerverwendbar u. aufbau community möglich)
	%software mit sinn (anwendung, welche einzigartig sind)
	%englishsprachig: da ich in dieser sprache programmiere und online mit anderen menschen mich damit austausche
		
\end{letter}
